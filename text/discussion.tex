The two-point energy calibration proved to be an effective method to calibrate a
spectrum. After cleaning the raw data to remove electronic noise and excluding
the 79 keV line for $^{133}$Ba, the calibrated data corresponded well with the actual
gamma-ray energies. There was a small amount of error between the gamma energies
and the photopeaks, but this is expected because I only used a two-point
energy calibration. Once more sources are added in to the calibration that span the
entire energy range, the photopeaks
should correspond better with their corresponding energies.
